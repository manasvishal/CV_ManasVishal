%% LyX 2.3.6.1 created this file.  For more info, see http://www.lyx.org/.
%% Do not edit unless you really know what you are doing.
\documentclass[10pt,english]{article}
\usepackage{fontspec}
\usepackage{geometry}
\geometry{verbose,tmargin=2cm,bmargin=2cm,lmargin=1cm,rmargin=1cm}
\usepackage{fancyhdr}
\pagestyle{fancy}
\usepackage{xcolor}
\usepackage{array}
\usepackage{longtable}
\usepackage{booktabs}
\PassOptionsToPackage{normalem}{ulem}
\usepackage{ulem}
\usepackage[unicode=true]
 {hyperref}

\makeatletter

%%%%%%%%%%%%%%%%%%%%%%%%%%%%%% LyX specific LaTeX commands.
%% Because html converters don't know tabularnewline
\providecommand{\tabularnewline}{\\}

\@ifundefined{date}{}{\date{}}
%%%%%%%%%%%%%%%%%%%%%%%%%%%%%% User specified LaTeX commands.

\definecolor{cyan}{RGB}{0,128,128}
\definecolor{mycolor}{RGB}{0,164,204}
\usepackage{fancyhdr}
\usepackage{colortbl}
\usepackage{soul}
\usepackage{datetime}

\usepackage{enumitem}
\setlist{nolistsep}

\makeatother

\usepackage{polyglossia}
\setdefaultlanguage[variant=american]{english}
\begin{document}

\chead{Curriculum Vitae}

\lhead{Manas Vishal}

\rhead{\textcolor{gray}{Last updated : \today}}

\textbf{\uline{Profile:}} Ph.D. student in Computational Sciences
at the University of Massachusetts Dartmouth.Broadly interested in
General Relatvitity and High Performance Computing. Currently working
as a Distinguished Doctoral Fellow in the Engineering and Applied
Sciences program with a focus on modeling gravitational waves from
extreme mass ratio black hole binaries using advanced numerical and
mathematical techniques.\vspace{0.5cm}

{\color{mycolor}\hrule height 1mm}%
\begin{flushleft}
\begin{longtable}[c]{>{\raggedright}p{4cm}>{\raggedright}p{14cm}}
\raggedright{}\textbf{\uline{Personal Information}}\textbf{ :}\linebreak{}
 & \textbf{Name: }\textbf{\textcolor{blue}{Manas Vishal}}\textbf{ }\\
\tabularnewline
 & \textcolor{cyan}{Email :} vishalmanas28@gmail.com, mvishal@umassd.edu\textcolor{cyan}{}\linebreak{}
\textcolor{cyan}{Gender :} Male | \textcolor{cyan}{Date of birth :}
28/10/1997 | \textcolor{cyan}{Nationality :} Indian\linebreak{}
\textcolor{cyan}{\href{http://manasvishal.github.io/}{http://manasvishal.github.io/}}\tabularnewline
\raggedright{}\textbf{\uline{Research Experience}}\textbf{ :}\linebreak{}
 & {\color{mycolor}\hrule height 1mm}%\linebreak{}
\tabularnewline
\textcolor{cyan}{2021 - current }\textbf{\textcolor{blue}{(Ph.D. project)}} & \textbf{\textcolor{blue}{EMRI surrogate modeling and late time Kerr
tails with spinning primary Black Hole using discontinuous Galerkin
method}}\textbf{}\linebreak{}
\textbf{\small{}Dr. Gaurav Khanna and Dr. Scott Field, UMass Dartmouth}\linebreak{}
It is need of the hour to have a gravitational wave template bank
for upcoming detectors. This project deals with one of the primary
sources of the upcoming space borne detector LISA which are extreme
mass ratio black hole binaries. I have written a very accurate Teukolsky
solver code using discontinuous Galerkin method to model the waveform
from EMRI system with a spinning primary.\linebreak{}
\tabularnewline
\textcolor{cyan}{June 27, 2020 - June 2021 }\textbf{\textcolor{blue}{(MS
project)}} & \noindent \textbf{\textcolor{blue}{Massless Scalar Waves in AdS Spacetime}}\textbf{
}\linebreak{}
\textbf{\small{}Dr. Rajesh Nayak, IISER Kolkata}\linebreak{}
It has been shown in recent works that the non-asymptotic AdS spacetime
eventually loses the curvature due to the formation of a black hole.
The boundary conditions for this system are complicated, and hence
very little work has been done on this topic computationally. I intend
to explore the stability of the AdS spacetime when perturbed by gravitational
waves. I am using the scalar perturbations to study the solution of
the massless scalar wave equation in this geometry. I have started
working on 4-dimensional spacetimes, but it can be extended to N-dimensions.\linebreak{}
\tabularnewline
\textcolor{cyan}{Aug. 2020 - Dec. 2020} & \textbf{\textcolor{blue}{Independent study on Magnetohydrodynamics
(MHD) and Fluid Dynamics}}\textbf{ }\linebreak{}
\textbf{\small{}Dr. Dibyendu Nandi, IISER Kolkata}\linebreak{}
The motivation for this independent study comes from General Relativistic
Magnetohydrodynamic simulation of Neutron Stars. I aim to teach myself
the concepts of Advection Equation and different types of flows in
fluid systems. I am acquiring skills in MHD simulations so that I
could learn the techniques of General Relativistic version of it.
\linebreak{}
\tabularnewline
\textcolor{cyan}{Dec. 2019 - March 2020} & \textbf{\textcolor{blue}{Finesse (Frequency domain INterfErometer
Simulation SoftwarE) Workshop and Hackathon}}\textbf{}\linebreak{}
\textbf{\small{}IUCAA Pune}\linebreak{}
Finesse is a sophisticated simulation package for modeling optics
and laser interferometers. This interferometer modeling software was
developed for the design of gravitational wave detectors, but is easy
to use for students with simpler lab-based setups as well. It includes
advanced features such as higher-order modes, quantum noise and radiation
pressure effects. I used this tool to model aLIGO detector for the
Hackathon.\linebreak{}
\tabularnewline
\textcolor{cyan}{May 15 - July 15, 2019} & \textbf{\textcolor{blue}{Reducing the flexing of the arms of LISA
- a space based Gravitational Wave detector}}\textbf{}\linebreak{}
\textbf{\small{}Dr. Rajesh Nayak, IISER Kolkata}\linebreak{}
Laser Interferometer Space Antenna (LISA) will be a spaced based gravitational
wave detector with an array of three spacecrafts in heliocentric orbit.
I developed a 3-body model to reduce the flexing in the arms of LISA
due to the breathing modes. This project helped me develop computational
skills, especially in using Python notebooks and theoretical concepts
like perturbation theory and celestial dynamics. The project ended
with me writing a Python code that could simulate the flexing in the
arms of LISA over its observing run.\linebreak{}
\tabularnewline
\textcolor{cyan}{Dec. 08 - Dec. 25, 2018} & \textbf{\textcolor{blue}{Deriving Geodesic equations for different
types of metrics}}\linebreak{}
\textbf{\small{}Dr. Naresh Dadhich, IUCAA Pune}\linebreak{}
A short reading project of “\textbf{Curved Space, curved Time,
and curved Space-Time in Schwarzschild geodetic geometry (\href{https://arxiv.org/abs/1812.03259}{arxiv:1812.03259})}.
I derived and calculated all the geodesic equations for different
types of metrics associated with Schwarzschild geometry, considering
space and time curvatures separately. The geodesic equations can then
be used to calculate the deflections due to 1 solar mass object and
it turned out to be exactly half of the “curved spacetime (1.75arcsec)”
in “curved space” and “curved time” metrics.\linebreak{}
\tabularnewline
\textcolor{cyan}{May 15 - July 15, 2018} & \textbf{\textcolor{blue}{Reading project in Cosmology, General Relativity
and Dark Matter}}\linebreak{}
\textbf{\small{}Dr. Subhadip Mitra, IIIT Hyderabad}\linebreak{}
A reading project in General Relativity, Cosmology and Dark Matter.
I studied General Relativity and Cosmology following textbooks by
Scott Dodelson, Bernard F. Schutz and James Hartle. This project helped
me gain the understanding of the basics of Tensor algebra and calculus,
General Relativity, Cosmology, and Dark Matter\linebreak{}
\tabularnewline
\textcolor{cyan}{May 20 - July 20, 2017} & \textbf{\textcolor{blue}{Quantum transport in mesoscopic system}}\linebreak{}
\textbf{\small{}Dr. Sourin Das, IISER Kolkata}\linebreak{}
I developed a Python code using Kwant module which could model topological
insulators in cubic and trapezoidal cubic geometry. This code was
used to simulate quantum hall effect in the mesoscopic systems by
attaching quantum gates to different sides of the cube, and measuring
the resistance offered by each side. \linebreak{}
\tabularnewline
\raggedright{}\textbf{\uline{Personal Skills}}\textbf{}\linebreak{}
 & {\color{mycolor}\hrule height 1mm}%\linebreak{}
\tabularnewline
\textcolor{cyan}{Digital Skills} & \textcolor{blue}{Programming Languages}:- C, Julia, Python (modules:
AstroPy, PyCBC, SciPy, NumPy, SymPy, pandas, Kwant, QuTip), R, SQL,
jQuery, HTML, PHP, \TeX\linebreak{}
\linebreak{}
\textcolor{blue}{Softwares} : - MATLAB, Mathematica, Origin, GnuPlot,
ImageJ, \LaTeX, Android Studio\linebreak{}
\tabularnewline
\textcolor{cyan}{Languages} & Hindi, English {[}TOEFL Score - Reading (21), Listening (25), Speaking
(22), Writing (21){]}\linebreak{}
\tabularnewline
\raggedright{}\textbf{\uline{Education and Training}}\textbf{:}\linebreak{}
 & {\color{mycolor}\hrule height 1mm}%\linebreak{}
\tabularnewline
\textcolor{cyan}{2021 - current} & \textbf{\textcolor{blue}{Ph.D., Computational Sciences}}\textbf{}\linebreak{}
\textbf{University of Massachusetts Dartmouth (U.S.A)}\linebreak{}
\textbf{Department}:- Engineering and Applied Sciences\linebreak{}
\linebreak{}
\tabularnewline
\textcolor{cyan}{2016 - 2021} & \textbf{\textcolor{blue}{5 Year BS-MS Dual Degree}}\textbf{}\linebreak{}
\textbf{Indian Institute of Science Education and Research, Kolkata(India)}\linebreak{}
\textbf{Department}:- Physics\linebreak{}
\textbf{CGPA:-} 8.4 \linebreak{}
\tabularnewline
\newpage
\raggedright{}\textbf{\uline{Teaching Assitantships}}\textbf{:}\linebreak{}
 & {\color{mycolor}\hrule height 1mm}%\linebreak{}
\tabularnewline
 & 1. Teaching Assistant for SS4101 (Space Astronomy) course offered
by Center of Excellence in Space Sciences India (CESSI), IISER Kolkata
to 4th year students in Autumn 2020.

2. Teaching Assistant for PH1201 (basic Electricity and Magnetism)
course offered by Department of Physical Sciences, IISER Kolkata at
freshman level in Spring 2020.\linebreak{}
\tabularnewline
\raggedright{}\textbf{\uline{Academic Awards}}\textbf{:}\linebreak{}
 & {\color{mycolor}\hrule height 1mm}%\linebreak{}
\tabularnewline
\textcolor{cyan}{2021 - current} & \textbf{\textcolor{blue}{\href{https://www.umassd.edu/graduate/fellowships-funding/distinguished-doctoral-fellowships/}{Distinguished Doctoral Fellowship}}},
offered by the University of Massachusetts Dartmouth to pursue my
research in black hole physics with Prof. Scott Field and Prof. Gaurav
Khanna. \linebreak{}
\tabularnewline
\textcolor{cyan}{Aug. 2017 - June 2021} & \textbf{\textcolor{blue}{KVPY }}Scholarship, India (offered by the
Department of Science and Technology, Government of India, to attract
exceptionally highly motivated students for pursuing basic science
courses and research career in science.)\linebreak{}
\tabularnewline
\textcolor{cyan}{Aug. 2016 - July 2017} & \textbf{\textcolor{blue}{INSPIRE}}\textbf{ (Innovation in Science
Pursuit for Inspired Research)} Fellowship (offered by the Department
of Science \& Technology to top\textbf{ 1\%} students to pursue a
career in science)\linebreak{}
\tabularnewline
\textcolor{cyan}{2013} & Awarded \textbf{Gold medal} for top performance in 10th standard.\linebreak{}
\tabularnewline
\textcolor{cyan}{Other Achievements} & 1. Selected for Super-30 in 2013 (a program which selects 30 talented
candidates among thousands of applications each year to prepare them
for \textbf{JEE,} an engineering entrance exam)\linebreak{}
2. Qualified several Olympiads organized by Science Olympiad Foundation\linebreak{}
3. Made an android application for Inquivesta, largest science fest
of India.\linebreak{}
\tabularnewline
\raggedright{}\textbf{\uline{Conferences and Workshops}}\textbf{
:}\linebreak{}
 & {\color{mycolor}\hrule height 1mm}%\linebreak{}
\tabularnewline
\textcolor{cyan}{Attended} & \begin{itemize}
\item Advances in Computational Relativity at ICERM, Brown University (September
9, 2020 - December 11, 2020)
\item North American Einstein Toolkit Workshop 2020 (August 3-7, 2020),
CCT, Louisiana State University (Virtual)
\item TCAN on Binary Neutron Stars Workshop 2020 (July 6-10, 2020) CCRG,
Rochester Institute of Technology (Virtual)
\item 23rd Capra Meeting on Radiation Reaction in General Relativity June
22-26, 2020) University of Texas at Austin (Virtual)
\item Cosmology Summer School 2020 (June 1-5, 2020), University of Michigan
(Virtual) 
\item BHPToolkit Spring 2020 workshop (May 25-27, 2020), Astronomical Institute
of the Czech Academy of Sciences (Virtual)
\item Applications of Data Science in Astrophysics and Gravitational Wave
Research (DSAP 2019) workshop held at IIIT Allahabad (November 1-3,
2019)
\item GPU based High Performance Computing workshop at IISER Kolkata (October
14-15, 2019)
\item Quantum Information and Quantum Technology (QIQT) 2019 at IISER Kolkata
(June 13-July 27, 2019).\linebreak{}
\end{itemize}
\tabularnewline
\textcolor{cyan}{Talks given} & \begin{itemize}
\item Presented my Ph.D. project at \href{https://meetings.aps.org/Meeting/APR22/Session/T16.4}{APS April meeting}
in New York City.
\item Presented my work over summer of 2020 at \href{http://cscvr1.umassd.edu/seminars.html}{UMassD Physics colloquium}
(September 3, 2020)
\item Gave an introductory talk on LISA and EMRI waveforms in \href{https://sites.google.com/view/surendrapadamata/mini-astro-workshop}{Mini-Astro Workshop}
(October 6, 2020)
\end{itemize}
\tabularnewline
\raggedright{}\textbf{\uline{References}}\textbf{:}\\
 & {\color{mycolor}\hrule height 1mm}%\\
\tabularnewline
\raggedright{} & \textbf{Available on request}\tabularnewline
\newpage
\raggedright{}\textbf{\uline{References}}\textbf{:}\\
 & {\color{mycolor}\hrule height 1mm}%\\
\tabularnewline
 & \textbf{Available on request}

\tabularnewline
\end{longtable}
\par\end{flushleft}
\end{document}
